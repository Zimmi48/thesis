\chapter*{Remerciements}

Je souhaite tout d'abord remercier Jean-R\'emy Falleri et Beno\^it Combemale d'avoir accept\'e de rapporter ma th\`ese peu de temps apr\`es m'avoir rencontr\'e, alors que je faisais mes premiers pas dans les \'ev\`enements%
\footnote{
    Journ\'ee GLE-RIMEL-LOUISE \`a Bordeaux en avril 2019, o\`u j'ai rencontr\'e Jean-R\'emy, et journ\'ees nationales du GDR GPL \`a Toulouse en juin 2019, o\`u j'ai rencontr\'e Beno\^it ainsi que de nombreux autres membres de l'\'equipe DiverSE.
}
de la communaut\'e fran\c{c}aise du g\'enie logiciel, au deuxi\`eme semestre de ma derni\`ere ann\'ee de th\`ese.

Ces remerciements s'\'etendent naturellement aux autres membres de mon jury, Anne Etien, rencontr\'ee \`a l'IRIF en mars 2019, puis \`a Toulouse et \`a Cleveland, et qui est devenue pour moi un point de rep\`ere dans la communaut\'e, Tom Mens, dont je n'ai commenc\'e \`a d\'ecouvrir les travaux qu'\`a la toute fin 2018, mais qui est sans doute l'auteur sur lequel je m'appuie le plus dans cette th\`ese, Pierre Courtieu, que j'ai bien plus c\^otoy\'e sur GitHub que dans le monde physique, malgr\'e notre proximit\'e g\'eographique, et Ralf Treinen, irr\'eductible repr\'esentant de l'\'etude des logiciels libres et des vrais syst\`emes logiciels \`a l'IRIF.

Je suis plus g\'en\'eralement reconnaissant pour le tr\`es bon accueil que j'ai re\c{c}u dans la communaut\'e du g\'enie logiciel, quand j'ai cherch\'e \`a m'y ins\'erer tardivement, et dans celle des langages de programmation et de l'informatique fondamentale, qui m'avait port\'e jusque-l\`a.
Je remercie en particulier Gabriel Radane de m'avoir fait d\'ecouvrir tardivement%
\footnote{
    En juin 2018, soit en fin de deuxi\`eme ann\'ee!
}
l'existence du GDR GPL.

La rigueur des travaux d'\'evaluation empirique dans cette th\`ese doit \'enorm\'ement \`a Annal\'i Casanueva Art\'is.
Notre amiti\'e est ancienne, mais notre collaboration scientifique remonte seulement \`a septembre 2018.
Moins de deux semaines avant de soumettre la premi\`ere version de l'article sur le \emph{bug tracker}, j'ai fait appel \`a elle%
\footnote{
    Ainsi qu'\`a Ambre Williams, que je remercie vivement \'egalement.
}
pour obtenir des conseils en analyse de donn\'ees.
Elle m'a non seulement pr\'esent\'e la m\'ethode d'analyse causale que j'allais utiliser \`a deux reprises dans cette th\`ese, mais elle m'a \'egalement form\'e en \'econom\'etrie et a grandement particip\'e \`a la maturation de cet article, devenu le n\^otre, jusqu'\`a la version qui a \'et\'e finalement publi\'ee \`a ICSME, et ce alors qu'elle-m\^eme doit avancer sa propre th\`ese sur un tout autre sujet.

L'aboutissement de cette th\`ese doit \'egalement beaucoup \`a Yann R\'egis-Gianas, co-directeur de la derni\`ere ann\'ee, qui a eu la gentillesse de me prendre sous son aile, et de m'aider \`a percer dans le nouveau domaine auquel je m'\'etais identifi\'e, alors que lui-m\^eme devait \'ecrire son HDR et assurer un grand nombre de responsabilit\'es par ailleurs.

Enfin, ne croyez pas que j'oublie Hugo Herbelin, mon directeur de th\`ese qui m'a accueilli et accompagn\'e dans le \sout{merveilleux} terrifiant monde de Coq, d\`es 2014--2015.
C'est gr\^ace aux nombreuses heures pass\'ees avec lui, pas seulement \`a discuter en rebondissant l'un l'autre en d'interminables digressions, mais aussi \`a coder ensemble un correctif ou une addition que nous venions d'imaginer, que j'ai commenc\'e \`a contribuer \`a Coq et \`a m'int\'eresser de pr\`es aux processus de d\'eveloppement.
Ses propres questionnements ont nourri mon approche.
Je le remercie \'egalement pour la grande libert\'e qu'il m'a laiss\'e, m\^eme lorsque je lui ai annonc\'e, en janvier 2018, c'est-\`a-dire en plein milieu de ma th\`ese, l'orientation que j'allais donner \`a celle-ci, r\'esolument empirique, et bien loin de tout ce \`a quoi il \'etait habitu\'e.

J'\'etends mes remerciements au reste des d\'eveloppeurs de Coq et son \'ecosyst\`eme, \`a commencer par Maxime D\'en\`es, avec qui j'ai collabor\'e sur le processus de sortie de nouvelles versions, Emilio Gallego Arias et Ga\"etan Gilbert, avec qui j'ai collabor\'e \`a la mise en place et \`a l'am\'elioration de notre syst\`eme d'int\'egration continue, Pierre Letouzey, qui m'a aid\'e lors de la migration du bug tracker, Guillaume Melquiond, Vincent Laporte et Pierre-Marie P\'edrot, qui ont servi de cobayes au syst\`eme de \emph{backporting} que j'avais mis en place, Matthieu Sozeau dont j'ai repris et jamais fini le projet de fusion de diff\'erentes tactiques d'automatisation, et Cyprien Mangin, qui m'avait pourtant aid\'e \`a surmonter certaines difficult\'es qui y \'etaient associ\'ees, Cl\'ement Pit-Claudel et Jim Fehrle, avec qui nous avons form\'e l'\'equipe de mainteneurs de la documentation, Pierre Cast\'eran, Karl Palmskog et Anton Trunov qui ont soutenu activement la cr\'eation de coq-community, mais aussi Tej Chajed, Guillaume Claret, Jason Gross, Arma\"el Gu\'eneau, Matej Ko\v{s}ik, Erik Martin-Dorel, Pierre Roux, Michael Soegtrop, Arnaud Spiwack, Nicolas Tabareau, Enrico Tassi, etc.
Remerciements particuliers \`a Yves Bertot, Assia Mahboubi et Cyril Cohen qui m'ont soutenu \`a divers moments o\`u j'en avais besoin.
Sans oublier les grands anciens, qui nous ont laiss\'e un objet d'\'etude extraordinaire, et les petits nouveaux, contributeurs occasionnels aux localisations g\'eographiques vari\'ees, trop nombreux pour \^etre cit\'es.

Je remercie tous les membres de l'IRIF, d'avoir cr\'e\'e un laboratoire chaleureux malgr\'e la couleur douteuse des peintures et la lumi\`ere des n\'eons.
J'ai eu beaucoup de plaisir \`a \^etre co-responsable du s\'eminaire doctorant, et je tiens \`a remercier mon pr\'ed\'ecesseur Cl\'ement, mes acolytes Laurent et Baptiste, et mes successeurs Chaitanya et Simon.
Merci aux doctorants des diff\'erentes g\'en\'erations (liste n\'ecessairement incompl\`ete)\,: Cyrille, \'Etienne, Jovana, Ludovic, Maxime, Pierre\,; Adrien, Jules, L\'eonard, R\'emi\,; Antoine, Axel, C\'edric, L\'eo, Nicolas, Victor, Zeinab\,; Alen, Farzad, F\'elix, Ga\"etan, Hugo, Kostia, Lo\"ic, etc.

Mon introduction \`a la recherche n'a pas commenc\'e par la th\`ese\,: je veux remercier tout particuli\`erement Tandy Warnow, qui m'a pr\'ec\'edemment introduit \`a un tout autre domaine, ainsi que les th\'esards de son groupe.
Je remercie \'egalement un sous-ensemble strict des profs qui m'ont conduit jusqu'\`a la th\`ese\,: j'ai \'et\'e marqu\'e par la passion de certains enseignants, \`a l'\'ecole primaire, au coll\`ege, au lyc\'ee, en pr\'epa, \`a l'ENS, etc.

Je remercie mes amis, connus \`a diff\'erentes \'epoques\,: Alexandre et Bruno\,; Luca et R\'emi\,; \'Edouard, Jean-Baptiste et Valentin\,; Arthur, Camille et Thomas, Damir, Pierre, Pierre-Antoine, Rapha\"el,%
\footnote{
    Techniquement connu \`a l'\'epoque pr\'ec\'edente.
}
R\'emi, S\'eb\,; Adrien, Alex, Anne-Sophie et Quentin, Antoine et Nadia, Catherine, Gaspard, Marie, Mayotte%
\footnote{
    Techniquement connue \`a l'\'epoque pr\'ec\'edente.
}
et Simon, Olivier et L\'eopoldine, Vincent\,; Ambre et Annal\'i, Basile, Damien, Florian, les deux Guillaume, Julie et Abel, Kenji, Ulysse et Ana\"elle et tous les autres anciens du C2\,;
les occupants de House of Commons, Austin, TX\,; Renaud\,; Max et Loriane, etc.

Je remercie ma famille, et tout particuli\`erement mes parents, Aline et Jean-Beno\^it, pour leur soutien ind\'efectible et crucial, ainsi que mon fr\`ere Jules, qui a su me poser les bonnes questions au bon moment et a eu une influence ind\'eniable sur ma th\`ese.
Enfin, je remercie C\'ecile pour toutes nos aventures partag\'ees.

\clearpage
