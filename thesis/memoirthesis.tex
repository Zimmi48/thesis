%\title{University of Bristol Thesis Template}
%\RequirePackage[l2tabu]{nag}		% Warns for incorrect (obsolete) LaTeX usage
%
%
% File: memoirthesis.tex
% Author: Victor Brena
% Description: Contains the thesis template using memoir class,
% which is mainly based on book class but permits better control of 
% chapter styles for example. This template is an adaptation and 
% modification of Oscar's.
% 
% Memoir is a flexible class for typesetting poetry, fiction, 
% non-fiction and mathematical works as books, reports, articles or
% manuscripts. CTAN repository is found at:
% http://www.ctan.org/tex-archive/macros/latex/contrib/memoir/
%
%
% UoB guidelines for thesis presentation were found at:
% http://www.bris.ac.uk/esu/pg/pgrcop11-12topic.pdf#page=49
%
% UoB guidelines:
%
% The dissertation must be printed on A4 white paper. Paper up to A3 may be used
% for maps, plans, diagrams and illustrative material. Pages (apart from the
% preliminary pages) should normally be double-sided.
%
% Memoir class loads useful packages by default (see manual).
\documentclass[a4paper,11pt,openbib]{memoir} %add 'draft' to turn draft option on (see below)
%


%%%%%%%%%%%%%%%%%%%%%%%%%%%%%%%%
%% Added packages

% Strikethrough
\usepackage[normalem]{ulem}

%% Subcaption
\usepackage{subcaption}
\captionsetup{compatibility=false}

\usepackage{fixmath} %bold math symbols

\usepackage{multirow}

%%Matrices avec indices visibles
\usepackage{blkarray} %%JOCO


\usepackage{algorithm}
\usepackage{algorithmic}

%%landscape
\usepackage{lscape}


%afficher ou pas les colonnes d'un tableau
\usepackage{array}
\newcolumntype{H}{>{\setbox0=\hbox\bgroup}c<{\egroup}@{}}

%aligner les virgules dans un tableau
\usepackage{siunitx}


\usepackage{hhline}

%enumerate without line spacing
\usepackage{enumitem}

\newenvironment{absolutelynopagebreak}
  {\par\nobreak\vfil\penalty0\vfilneg
   \vtop\bgroup}
  {\par\xdef\tpd{\the\prevdepth}\egroup
   \prevdepth=\tpd}
   
%%%%%%%%%%%%%%%%%%%%%%%%%%%%%%%%%


%
% Adding metadata:
\usepackage{datetime}
\usepackage{ifpdf}
\ifpdf
\pdfinfo{
   /Author (Th\'eo Zimmermann)
   /Title (PhD Thesis)
%   /Keywords (One; Two;Three)
   /CreationDate (D:\pdfdate)
}
\fi
% When draft option is on. 
\ifdraftdoc 
	\usepackage{draftwatermark}				%Sets watermarks up.
	\SetWatermarkScale{0.3}
	\SetWatermarkText{\bf Draft: \today}
\fi
%
% Declare figure/table as a subfloat.
\newsubfloat{figure}
\newsubfloat{table}
% Better page layout for A4 paper, see memoir manual.
\settrimmedsize{297mm}{210mm}{*}
\setlength{\trimtop}{0pt} 
\setlength{\trimedge}{\stockwidth} 
\addtolength{\trimedge}{-\paperwidth} 
\settypeblocksize{634pt}{448.13pt}{*} 
\setulmargins{4cm}{*}{*} 
\setlrmargins{*}{*}{1.5} 
\setmarginnotes{17pt}{51pt}{\onelineskip} 
\setheadfoot{\onelineskip}{2\onelineskip} 
\setheaderspaces{*}{2\onelineskip}{*} 
\checkandfixthelayout
%
\frenchspacing
% Font with math support: New Century Schoolbook
\usepackage{fouriernc}
\usepackage[T1]{fontenc}
%
% UoB guidelines:
%
% Text should be in double or 1.5 line spacing, and font size should be
% chosen to ensure clarity and legibility for the main text and for any
% quotations and footnotes. Margins should allow for eventual hard binding.
%
% Note: This is automatically set by memoir class. Nevertheless \OnehalfSpacing 
% enables double spacing but leaves single spaced for captions for instance. 
\OnehalfSpacing 
%
% Sets numbering division level
\setsecnumdepth{subsection} 
\maxsecnumdepth{subsubsection}
%
% Chapter style (taken and slightly modified from Lars Madsen Memoir Chapter 
% Styles document
\usepackage{calc,soul,fourier}
\makeatletter 
\newlength\dlf@normtxtw 
\setlength\dlf@normtxtw{\textwidth} 
\newsavebox{\feline@chapter} 
\newcommand\feline@chapter@marker[1][4cm]{%
	\sbox\feline@chapter{% 
		\resizebox{!}{#1}{\fboxsep=1pt%
			\colorbox{black}{\color{white}\thechapter}% 
		}}%
		\rotatebox{90}{% 
			\resizebox{%
				\heightof{\usebox{\feline@chapter}}+\depthof{\usebox{\feline@chapter}}}% 
			{!}{\scshape\so\@chapapp}}\quad%
		\raisebox{\depthof{\usebox{\feline@chapter}}}{\usebox{\feline@chapter}}%
} 
\newcommand\feline@chm[1][4cm]{%
	\sbox\feline@chapter{\feline@chapter@marker[#1]}% 
	\makebox[0pt][c]{% aka \rlap
		\makebox[1cm][r]{\usebox\feline@chapter}%
	}}
\makechapterstyle{daleifmodif}{
	\renewcommand\chapnamefont{\normalfont\Large\scshape\so} 
	\renewcommand\chaptitlefont{\normalfont\huge\bfseries\scshape} 
	\renewcommand\chapternamenum{} \renewcommand\printchaptername{} 
	\renewcommand\printchapternum{\null\hspace{2.5cm}\feline@chm[2.5cm]}%\renewcommand\afterchapternum{\par\vskip\midchapskip} 	
	\renewcommand\printchaptertitle[1]{\color{black}\chaptitlefont ##1\par}
} 
\makeatother 
\chapterstyle{daleifmodif}

\renewcommand\partnamefont{\normalfont\Large\scshape\so} 
\renewcommand\parttitlefont{\normalfont\Huge\bfseries\scshape}  
\renewcommand\printpartnum{\hspace{0.5cm}\thepart}

%\renewcommand\sectionnamefont{\normalfont\scshape\so} 

%
% UoB guidelines:
%
% The pages should be numbered consecutively at the bottom centre of the
% page.
\makepagestyle{myvf} 
\makeoddfoot{myvf}{}{\thepage}{} 
\makeevenfoot{myvf}{}{\thepage}{} 
\makeheadrule{myvf}{\textwidth}{\normalrulethickness} 
\makeevenhead{myvf}{\small\textsc{\leftmark}}{}{} 
\makeoddhead{myvf}{}{}{\small\textsc{\rightmark}}
\pagestyle{myvf}
%
% Oscar's command (it works):
% Fills blank pages until next odd-numbered page. Used to emulate single-sided
% frontmatter. This will work for title, abstract and declaration. Though the
% contents sections will each start on an odd-numbered page they will
% spill over onto the even-numbered pages if extending beyond one page
% (hopefully, this is ok).
\newcommand{\clearemptydoublepage}{\newpage{\thispagestyle{empty}\cleardoublepage}}
%
%
% Creates indexes for Table of Contents, List of Figures, List of Tables and Index
\makeindex
% \printglossaries below creates a list of abbreviations. \gls and related
% commands are then used throughout the text, so that latex can automatically
% keep track of which abbreviations have already been defined in the text.
%
% The import command enables each chapter tex file to use relative paths when
% accessing supplementary files. For example, to include
% chapters/brewing/images/figure1.png from chapters/brewing/brewing.tex we can
% use
% \includegraphics{images/figure1}
% instead of
% \includegraphics{chapters/brewing/images/figure1}
\usepackage{import}

% Add other packages needed for chapters here. For example:
%\usepackage{lipsum}					%Needed to create dummy text
\usepackage{amsfonts} 					%Calls Amer. Math. Soc. (AMS) fonts
\usepackage[centertags]{amsmath}			%Writes maths centred down
%\usepackage{stmaryrd}					%New AMS symbols
\usepackage{amssymb}					%Calls AMS symbols
\usepackage{amsthm}					%Calls AMS theorem environment
\usepackage{newlfont}					%Helpful package for fonts and symbols
%\usepackage{layouts}					%Layout diagrams
\usepackage{graphicx}					%Calls figure environment
%\usepackage{longtable,rotating}			%Long tab environments including rotation. 
%\usepackage[applemac]{inputenc}			%Needed to encode non-english characters 
									%directly for mac
%\usepackage{colortbl}					%Makes coloured tables
%\usepackage{wasysym}					%More math symbols
%\usepackage{mathrsfs}					%Even more math symbols
\usepackage{float}						%Helps to place figures, tables, etc. 
%\usepackage{verbatim}					%Permits pre-formated text insertion
\usepackage{upgreek }					%Calls other kind of greek alphabet
\usepackage{latexsym}					%Extra symbols
\usepackage[square,numbers,
		     sort&compress]{natbib}		%Calls bibliography commands 
\usepackage{url}						%Supports url commands
%\usepackage{etex}						%eTeXÕs extended support for counters
%\usepackage{fixltx2e}					%Eliminates some in felicities of the 
									%original LaTeX kernel
\usepackage[english]{babel}		%For languages characters and hyphenation
\usepackage{color}                    				%Creates coloured text and background
\usepackage[colorlinks=true,
		     allcolors=black,backref]{hyperref}              %Creates hyperlinks in cross references
\usepackage{memhfixc}					%Must be used on memoir document 
									%class after hyperref
%\usepackage{enumerate}					%For enumeration counter
%\usepackage{footnote}					%For footnotes
\usepackage{microtype}					%Makes pdf look better.
%\usepackage{rotfloat}					%For rotating and float environments as tables, 
									%figures, etc. 
%\usepackage{alltt}						%LaTeX commands are not disabled in 
									%verbatim-like environment
\usepackage[version=0.96]{pgf}			%PGF/TikZ is a tandem of languages for producing vector graphics from a 
\usepackage{tikz}						%geometric/algebraic description.
\usetikzlibrary{arrows,shapes,snakes,
		       automata,backgrounds,
		       petri,topaths,calc}				%To use diverse features from tikz		
%							
%Reduce widows  (the last line of a paragraph at the start of a page) and orphans 
% (the first line of paragraph at the end of a page)
\widowpenalty=1000
\clubpenalty=1000
%
% New command definitions for my thesis
%

\newcommand{\pgftextcircled}[1]{                                                                    %Defines encircled text
    \setbox0=\hbox{#1}%
    \dimen0\wd0%
    \divide\dimen0 by 2%
    \begin{tikzpicture}[baseline=(a.base)]%
        \useasboundingbox (-\the\dimen0,0pt) rectangle (\the\dimen0,1pt);
        \node[circle,draw,outer sep=0pt,inner sep=0.1ex] (a) {#1};
    \end{tikzpicture}
}
                            %Defines arctanh
%Change tombstone symbol
\newcommand{\blackged}{\hfill$\blacksquare$}
\newcommand{\whiteged}{\hfill$\square$}
\newcounter{proofcount}
\renewenvironment{proof}[1][\proofname.]{\par
 \ifnum \theproofcount>0 \pushQED{\whiteged} \else \pushQED{\blackged} \fi%
 \refstepcounter{proofcount}
 \normalfont 
 \trivlist
 \item[\hskip\labelsep
       \itshape
   {\bf\em #1}]\ignorespaces
}{%
 \addtocounter{proofcount}{-1}
 \popQED\endtrivlist
}
%
%
% New definition of square root:
% it renames \sqrt as \oldsqrt
\let\oldsqrt\sqrt
% it defines the new \sqrt in terms of the old one
\def\sqrt{\mathpalette\DHLhksqrt}
\def\DHLhksqrt#1#2{%
\setbox0=\hbox{$#1\oldsqrt{#2\,}$}\dimen0=\ht0
\advance\dimen0-0.2\ht0
\setbox2=\hbox{\vrule height\ht0 depth -\dimen0}%
{\box0\lower0.4pt\box2}}
%
% My caption style
\newcommand{\mycaption}[2][\@empty]{
	\captionnamefont{\scshape} 
	\changecaptionwidth
	\captionwidth{0.9\linewidth}
	\captiondelim{.\:} 
	\indentcaption{0.75cm}
	\captionstyle[\centering]{}
	\setlength{\belowcaptionskip}{10pt}
	\ifx \@empty#1 \caption{#2}\else \caption[#1]{#2}
}
%
% My subcaption style
\newcommand{\mysubcaption}[2][\@empty]{
	\subcaptionsize{\small}
	\hangsubcaption
	\subcaptionlabelfont{\rmfamily}
	\sidecapstyle{\raggedright}
	\setlength{\belowcaptionskip}{10pt}
	\ifx \@empty#1 \subcaption{#2}\else \subcaption[#1]{#2}
}
%
%An initial of the very first character of the content
\usepackage{lettrine}
\newcommand{\initial}[1]{%
	\lettrine[lines=3,lhang=0.33,nindent=0em]{
		\color{gray}
     		{\textsc{#1}}}{}}
%
% Theorem styles used in my thesis
%
\theoremstyle{plain}
\newtheorem{theorem}{Theorem}[chapter]
\theoremstyle{plain}
\newtheorem{proposition}{Proposition}[chapter]
\theoremstyle{plain}
\theoremstyle{definition}
\newtheorem{definition}{Definition}[chapter]
\theoremstyle{plain}
\newtheorem{lemma}{Lemma}[chapter]
\theoremstyle{plain}
\newtheorem{corollary}{Corollary}[chapter]
\theoremstyle{plain}
\newtheorem{result}{Result}[chapter]
\theoremstyle{plain}
\newtheorem{example}{Example}[chapter]
\theoremstyle{plain}
\newtheorem{property}{Property}[chapter]
\theoremstyle{plain}
\newtheorem{remark}{Remark}[chapter]

\theoremstyle{plain} % just in case the style had changed
\newcommand{\thistheoremname}{}
\newtheorem*{genericthm*}{\thistheoremname}
\newenvironment{namedthm*}[1]
  {\renewcommand{\thistheoremname}{#1}%
   \begin{genericthm*}}
  {\end{genericthm*}}
%
%




%%%%%%%%%%%%%%%%%%%%%%%%%%%%%%%%
%% PACKAGE RAJOUTES
% on est obligé de le mettre là, c'est incompatible avec un autre truc


%équations numérotées et entre accolades
\usepackage[subnum]{cases}



%%%%%%%%%%%%%%%%%%%%%%%%%%%%%%%%%

\begin{document}
\raggedbottom % pour éviter que il y ait des espaces énormes quand il saute de page à cause d'une formulation

% UoB guidlines:
%
% Preliminary pages
% 
% The five preliminary pages must be the Title Page, Abstract, Dedication
% and Acknowledgements, Author's Declaration and Table of Contents.
% These should be single-sided.
% 
% Table of contents, list of tables and illustrative material
% 
% The table of contents must list, with page numbers, all chapters,
 % sections and subsections, the list of references, bibliography, list of
% abbreviations and appendices. The list of tables and illustrations
% should follow the table of contents, listing with page numbers the
% tables, photographs, diagrams, etc., in the order in which they appear
% in the text.
% 
\frontmatter
\pagenumbering{roman}
%
%
\begin{titlingpage}
\begin{SingleSpace}
\calccentering{\unitlength} 
%\begin{adjustwidth*}{\unitlength}{-\unitlength}


\includegraphics[height=1.8cm]{frontmatter/uni_paris.jpg}  \hfill
\includegraphics[height=1.8cm]{frontmatter/irif.eps} \hfill
\includegraphics[height=1.8cm]{frontmatter/inria.png}

\begin{center}

\textsc{\'Ecole doctorale 386 : Sciences math\'ematiques de Paris centre}\\
\vspace{2mm}
\textsc{Universit\'e de Paris, IRIF}\\
\vspace{2mm}
\textsc{Inria, \'Equipe-projet $\pi.r^2$}\\

\vspace{4mm}

\textsc{\Large \textbf{Th\`ese de doctorat en informatique}}\\

\vspace*{2mm}

\rule[0.5ex]{\linewidth}{2pt}\vspace*{-\baselineskip}\vspace*{3.2pt}
\rule[0.5ex]{\linewidth}{1pt}\\[\baselineskip]
\vspace{-0.35cm}
{\huge Challenges in the collaborative evolution of a proof language and its ecosystem }\\
\rule[0.5ex]{\linewidth}{1pt}\vspace*{-\baselineskip}\vspace*{3.4pt}
\rule[0.5ex]{\linewidth}{2pt}\\

\vspace{2mm}

{\large Par } {\large\textsc{\textbf{Th\'eo Zimmermann}}}\\

\vspace{5mm}

{\large Dirig\'ee par }\\

\vspace{3mm}

\begin{tabular}{lr}
  \textbf{Hugo {\scshape Herbelin}} & \\
  Directeur de recherche Inria \`a l'Universit\'e de Paris \hspace{17mm}
  & Directeur de th\`ese  \\
  & \\
  \textbf{Yann {\scshape R\'egis-Gianas}} & \\
  Ma\^itre de conf\'erence HDR
  \`a l'Universit\'e de Paris
  & Co-directeur \\
\end{tabular}

\vspace{6mm}

{\large Pr\'esent\'ee et soutenue publiquement le 12 d\'ecembre 2019, \\
 devant un jury compos\'e de : }\\

\vspace{4mm}

\begin{tabular}{lr}
  \textbf{Beno\^it {\scshape Combemale}} & \\
  Professeur \`a l'Universit\'e Toulouse-Jean Jaur\`es
  & Rapporteur \\
  & \\
  \textbf{Jean-R\'emy {\scshape Falleri}} & \\
  Ma\^itre de conf\'erence HDR \`a l'Universit\'e de Bordeaux
  & Rapporteur \\
  & \\
  \textbf{Pierre {\scshape Courtieu}} & \\
  Ma\^itre de conf\'erence au Conservatoire national des arts et m\'etiers
  & Examinateur \\
  & \\
  \textbf{Anne {\scshape Etien}} & \\
  Ma\^itre de conf\'erence HDR \`a l'Universit\'e de Lille
  & Examinatrice \\
  & \\
  \textbf{Tom {\scshape Mens}} & \\
  Professeur \`a l'Universit\'e de Mons
  & Examinateur \\
  & \\
  \textbf{Ralf {\scshape Treinen}} & \\
  Professeur \`a l'Universit\'e de Paris
  & Examinateur \\
  & \\
\end{tabular}

%\vspace{9mm}
%{\large\textsc{April 2013}}
%\vspace{12mm}
\end{center}
\begin{flushright}
%{\small Word count: ten thousand and four}
\end{flushright}
%\end{adjustwidth*}
\end{SingleSpace}
\end{titlingpage}

\clearemptydoublepage

%
\begin{minipage}[t][0.5\textheight][t]{\textwidth}
    
\section*{Abstract}
\begin{SingleSpace}

In this thesis, I present the application of software engineering methods and knowledge to the development, maintenance, and evolution of Coq ---an interactive proof assistant based on type theory--- and its package ecosystem.
Coq has been developed at Inria since 1984, but has only more recently seen a surge in its user base, which leads to greater concerns about its maintainability, and the involvement of external contributors in the evolution of both Coq and its ecosystem of plugins and libraries.

Recent years have seen important changes in the development processes of Coq, of which I have been a witness and an actor (adoption of GitHub as a development platform, first for its pull request mechanism, then for its bug tracker, adoption of continuous integration, switch to shorter release cycles, increased involvement of external contributors in the open source development and maintenance process).
The contributions of this thesis include a historical description of these changes, the refinement of existing processes, the design of new processes, the design and implementation of new tools to help the application of these processes, and the validation of these changes through rigorous empirical evaluation.

Involving external contributors is also very useful at the level of the package ecosystem.
This thesis additionally contains an analysis of package distribution methods, and a focus on the problem of the long-term maintenance of single-maintainer packages.
\end{SingleSpace}

\vspace{3mm}

\textbf{Keywords:} empirical software engineering, proof assistant, Coq, open source, open collaboration, software maintenance and evolution, GitHub, package ecosystem.

\end{minipage} \\
    
    
\begin{minipage}[b][0.49\textheight][t]{\textwidth}
%\vspace{10mm}

\section*{R\'esum\'e}
\begin{SingleSpace}

Dans cette th\`ese, je pr\'esente l'application de m\'ethodes et de connaissances en g\'enie logiciel au d\'eveloppement, \`a la maintenance et \`a l'\'evolution de Coq ---un assistant de preuve interactif bas\'e sur la th\'eorie des types--- et de son \'ecosyst\`eme de paquets.
Coq est d\'evelopp\'e chez Inria depuis 1984, mais sa base d'utilisateurs n'a cess\'e de s'agrandir, ce qui suscite d\'esormais une attention renforc\'ee quant \`a sa maintenabilit\'e et \`a la participation de contributeurs externes \`a son \'evolution et \`a celle de son \'ecosyst\`eme de plugins et de biblioth\`eques.

D'importants changements ont eu lieu ces derni\`eres ann\'ees dans les processus de d\'eveloppement de Coq, dont j'ai \'et\'e à la fois un t\'emoin et un acteur (adoption de GitHub en tant que plate-forme de d\'eveloppement, tout d'abord pour son m\'ecanisme de \emph{pull request}, puis pour son syst\`eme de tickets, adoption de l'intégration continue, passage \`a des cycles de sortie de nouvelles versions plus courts, implication accrue de contributeurs externes dans les processus de d\'eveloppement et de maintenance \emph{open source}).
Les contributions de cette th\`ese incluent une description historique de ces changements, le raffinement des processus existants et la conception de nouveaux processus, la conception et la mise en {\oe}uvre de nouveaux outils facilitant l'application de ces processus, et la validation de ces changements par le biais d'\'evaluations empiriques rigoureuses.

L'implication de contributeurs externes est \'egalement tr\`es utile au niveau de l'\'ecosyst\`eme de paquets.
Cette th\`ese contient en outre une analyse des m\'ethodes de distribution de paquets et du probl\`eme sp\'ecifique de la maintenance \`a long terme des paquets ayant un seul responsable.

\vspace{3mm}

\textbf{Mots-cl\'es :} g\'enie logiciel empirique, assistant de preuve, Coq, open source, collaboration ouverte, maintenance et \'evolution du logiciel, GitHub, \'ecosyst\`eme de paquets.

\end{SingleSpace}
\end{minipage}

\clearpage

\chapter*{Extended abstract}

Research on programming languages and underlying logical foundations has led to the design of programming languages with incredibly powerful type systems that give much stronger guarantees to programmers.
During the same time, research on software engineering has led to technical and methodological advances that have made programmers incredibly more productive.
Unfortunately, there have been too few bridges between the two domains. Some of the most advanced programming languages have been developed in academic teams with little knowledge on software engineering, and even less regarding recent research progress.
In this thesis, I present the application of software engineering methods and knowledge to the development, maintenance, and evolution of Coq ---an interactive proof assistant that may also be viewed as a programming language with a very strong type system (based on type theory)--- and its package ecosystem.
Coq has been developed at Inria since 1984, but has only more recently seen a surge in its user base, which leads to much stronger concerns about its maintainability, and the involvement of external contributors in the evolution of both Coq as a proof language, and its ecosystem of plugins and libraries.

Recent years have seen important changes in the development processes of Coq, of which I have been a witness and an actor.
The contributions of this thesis include a historical description of these changes, the refinement of existing processes, and the design of new ones, the design and implementation of new tools to help the application of these processes, and the validation of these changes through empirical evaluation.

The switch from a push-based to a pull-based development model was first motivated by the opening to new contributors, but has also affected the quality of integrated changes.
I present a historical overview of the adoption of pull-based development, and the changes that we implemented following this switch, including the systematic testing of external reverse dependencies through continuous integration, the use of labels, pull request templates, and the involvement of external contributors in the pull request integration process.
Other changes include a switch of bug tracker, and the adoption of shorter release cycles.
Each of these changes had associated challenges, which I identified and contributed to addressing.
For instance, switching bug trackers required migrating preexisting issues and preserving meta-data as much as possible, which I achieved by reusing and adapting an existing migration tool.
Switching to shorter release cycles required inventing a process and implementing associated tooling to manage efficiently the backporting process and the preparation of release notes.
The solutions that I describe would be easy to apply beyond the Coq project.

To empirically validate some actions that were taken, I have used a technique for causality analysis, imported from econometrics, on mined software repository data.
I show that the switch of bug trackers, from Bugzilla to GitHub, resulted in more activity by core developers, and more external participation in discussions on bug reports.
Using the same technique, I show that the introduction of a pull request template resulted in a higher proportion of pull requests including documentation, tests, and a changelog entry.
This technique is presented in much detail, and could easily be applied to many more studies based on mined software repository data, in the context of empirical software engineering, to obtain more compelling results that go beyond the identification of correlations.

Finally, involving external contributors is also very useful at the level of the package ecosystem.
This thesis contains an analysis of package distribution methods, and a focus on the problem of the long-term maintenance of single-maintainer packages.
I identified an emerging model of community organizations addressing this problem, which I present both abstractly and with specific examples.
I founded an organization based on this model in the context of the Coq package ecosystem, and it is already quite successful.

\clearemptydoublepage
%
\chapter*{Remerciements}

Je souhaite tout d'abord remercier Jean-R\'emy Falleri et Beno\^it Combemale d'avoir accept\'e de rapporter ma th\`ese peu de temps apr\`es m'avoir rencontr\'e, alors que je faisais mes premiers pas dans les \'ev\`enements%
\footnote{
    Journ\'ee GLE-RIMEL-LOUISE \`a Bordeaux en avril 2019, o\`u j'ai rencontr\'e Jean-R\'emy, et journ\'ees nationales du GDR GPL \`a Toulouse en juin 2019, o\`u j'ai rencontr\'e Beno\^it ainsi que de nombreux autres membres de l'\'equipe DiverSE.
}
de la communaut\'e fran\c{c}aise du g\'enie logiciel, au deuxi\`eme semestre de ma derni\`ere ann\'ee de th\`ese.

Ces remerciements s'\'etendent naturellement aux autres membres de mon jury, Anne Etien, rencontr\'ee \`a l'IRIF en mars 2019, puis \`a Toulouse et \`a Cleveland, et qui est devenue pour moi un point de rep\`ere dans la communaut\'e, Tom Mens, dont je n'ai commenc\'e \`a d\'ecouvrir les travaux qu'\`a la toute fin 2018, mais qui est sans doute l'auteur sur lequel je m'appuie le plus dans cette th\`ese, Pierre Courtieu, que j'ai bien plus c\^otoy\'e sur GitHub que dans le monde physique, malgr\'e notre proximit\'e g\'eographique, et Ralf Treinen, irr\'eductible repr\'esentant de l'\'etude des logiciels libres et des vrais syst\`emes logiciels \`a l'IRIF.

Je suis plus g\'en\'eralement reconnaissant pour le tr\`es bon accueil que j'ai re\c{c}u dans la communaut\'e du g\'enie logiciel, quand j'ai cherch\'e \`a m'y ins\'erer tardivement, et dans celle des langages de programmation et de l'informatique fondamentale, qui m'avait port\'e jusque-l\`a.
Je remercie en particulier Gabriel Radane de m'avoir fait d\'ecouvrir tardivement%
\footnote{
    En juin 2018, soit en fin de deuxi\`eme ann\'ee!
}
l'existence du GDR GPL.

La rigueur des travaux d'\'evaluation empirique dans cette th\`ese doit \'enorm\'ement \`a Annal\'i Casanueva Art\'is.
Notre amiti\'e est ancienne, mais notre collaboration scientifique remonte seulement \`a septembre 2018.
Moins de deux semaines avant de soumettre la premi\`ere version de l'article sur le \emph{bug tracker}, j'ai fait appel \`a elle%
\footnote{
    Ainsi qu'\`a Ambre Williams, que je remercie vivement \'egalement.
}
pour obtenir des conseils en analyse de donn\'ees.
Elle m'a non seulement pr\'esent\'e la m\'ethode d'analyse causale que j'allais utiliser \`a deux reprises dans cette th\`ese, mais elle m'a \'egalement form\'e en \'econom\'etrie et a grandement particip\'e \`a la maturation de cet article, devenu le n\^otre, jusqu'\`a la version qui a \'et\'e finalement publi\'ee \`a ICSME, et ce alors qu'elle-m\^eme doit avancer sa propre th\`ese sur un tout autre sujet.

L'aboutissement de cette th\`ese doit \'egalement beaucoup \`a Yann R\'egis-Gianas, co-directeur de la derni\`ere ann\'ee, qui a eu la gentillesse de me prendre sous son aile, et de m'aider \`a percer dans le nouveau domaine auquel je m'\'etais identifi\'e, alors que lui-m\^eme devait \'ecrire son HDR et assurer un grand nombre de responsabilit\'es par ailleurs.

Enfin, ne croyez pas que j'oublie Hugo Herbelin, mon directeur de th\`ese qui m'a accueilli et accompagn\'e dans le \sout{merveilleux} terrifiant monde de Coq, d\`es 2014--2015.
C'est gr\^ace aux nombreuses heures pass\'ees avec lui, pas seulement \`a discuter en rebondissant l'un l'autre en d'interminables digressions, mais aussi \`a coder ensemble un correctif ou une addition que nous venions d'imaginer, que j'ai commenc\'e \`a contribuer \`a Coq et \`a m'int\'eresser de pr\`es aux processus de d\'eveloppement.
Ses propres questionnements ont nourri mon approche.
Je le remercie \'egalement pour la grande libert\'e qu'il m'a laiss\'e, m\^eme lorsque je lui ai annonc\'e, en janvier 2018, c'est-\`a-dire en plein milieu de ma th\`ese, l'orientation que j'allais donner \`a celle-ci, r\'esolument empirique, et bien loin de tout ce \`a quoi il \'etait habitu\'e.

J'\'etends mes remerciements au reste des d\'eveloppeurs de Coq et son \'ecosyst\`eme, \`a commencer par Maxime D\'en\`es, avec qui j'ai collabor\'e sur le processus de sortie de nouvelles versions, Emilio Gallego Arias et Ga\"etan Gilbert, avec qui j'ai collabor\'e \`a la mise en place et \`a l'am\'elioration de notre syst\`eme d'int\'egration continue, Pierre Letouzey, qui m'a aid\'e lors de la migration du bug tracker, Guillaume Melquiond, Vincent Laporte et Pierre-Marie P\'edrot, qui ont servi de cobayes au syst\`eme de \emph{backporting} que j'avais mis en place, Matthieu Sozeau dont j'ai repris et jamais fini le projet de fusion de diff\'erentes tactiques d'automatisation, et Cyprien Mangin, qui m'avait pourtant aid\'e \`a surmonter certaines difficult\'es qui y \'etaient associ\'ees, Cl\'ement Pit-Claudel et Jim Fehrle, avec qui nous avons form\'e l'\'equipe de mainteneurs de la documentation, Pierre Cast\'eran, Karl Palmskog et Anton Trunov qui ont soutenu activement la cr\'eation de coq-community, mais aussi Tej Chajed, Guillaume Claret, Jason Gross, Arma\"el Gu\'eneau, Matej Ko\v{s}ik, Erik Martin-Dorel, Pierre Roux, Michael Soegtrop, Arnaud Spiwack, Nicolas Tabareau, Enrico Tassi, etc.
Remerciements particuliers \`a Yves Bertot, Assia Mahboubi et Cyril Cohen qui m'ont soutenu \`a divers moments o\`u j'en avais besoin.
Sans oublier les grands anciens, qui nous ont laiss\'e un objet d'\'etude extraordinaire, et les petits nouveaux, contributeurs occasionnels aux localisations g\'eographiques vari\'ees, trop nombreux pour \^etre cit\'es.

Je remercie tous les membres de l'IRIF, d'avoir cr\'e\'e un laboratoire chaleureux malgr\'e la couleur douteuse des peintures et la lumi\`ere des n\'eons.
J'ai eu beaucoup de plaisir \`a \^etre co-responsable du s\'eminaire doctorant, et je tiens \`a remercier mon pr\'ed\'ecesseur Cl\'ement, mes acolytes Laurent et Baptiste, et mes successeurs Chaitanya et Simon.
Merci aux doctorants des diff\'erentes g\'en\'erations (liste n\'ecessairement incompl\`ete)\,: Cyrille, \'Etienne, Jovana, Ludovic, Maxime, Pierre\,; Adrien, Jules, L\'eonard, R\'emi\,; Antoine, Axel, C\'edric, L\'eo, Nicolas, Victor, Zeinab\,; Alen, Farzad, F\'elix, Ga\"etan, Hugo, Kostia, Lo\"ic, etc.

Mon introduction \`a la recherche n'a pas commenc\'e par la th\`ese\,: je veux remercier tout particuli\`erement Tandy Warnow, qui m'a pr\'ec\'edemment introduit \`a un tout autre domaine, ainsi que les th\'esards de son groupe.
Je remercie \'egalement un sous-ensemble strict des profs qui m'ont conduit jusqu'\`a la th\`ese\,: j'ai \'et\'e marqu\'e par la passion de certains enseignants, \`a l'\'ecole primaire, au coll\`ege, au lyc\'ee, en pr\'epa, \`a l'ENS, etc.

Je remercie mes amis, connus \`a diff\'erentes \'epoques\,: Alexandre et Bruno\,; Luca et R\'emi\,; \'Edouard, Jean-Baptiste et Valentin\,; Arthur, Camille et Thomas, Damir, Pierre, Pierre-Antoine, Rapha\"el,%
\footnote{
    Techniquement connu \`a l'\'epoque pr\'ec\'edente.
}
R\'emi, S\'eb\,; Adrien, Alex, Anne-Sophie et Quentin, Antoine et Nadia, Catherine, Gaspard, Marie, Mayotte%
\footnote{
    Techniquement connue \`a l'\'epoque pr\'ec\'edente.
}
et Simon, Olivier et L\'eopoldine, Vincent\,; Ambre et Annal\'i, Basile, Damien, Florian, les deux Guillaume, Julie et Abel, Kenji, Ulysse et Ana\"elle et tous les autres anciens du C2\,;
les occupants de House of Commons, Austin, TX\,; Renaud\,; Max et Loriane, etc.

Je remercie ma famille, et tout particuli\`erement mes parents, Aline et Jean-Beno\^it, pour leur soutien ind\'efectible et crucial, ainsi que mon fr\`ere Jules, qui a su me poser les bonnes questions au bon moment et a eu une influence ind\'eniable sur ma th\`ese.
Enfin, je remercie C\'ecile pour toutes nos aventures partag\'ees.

\clearpage

\clearemptydoublepage
%


%%%%% TABLE OF CONTENTS
\renewcommand{\contentsname}{Table of Contents}
\maxtocdepth{subsection}
\tableofcontents*
\addtocontents{toc}{\par\nobreak \mbox{}\hfill{\bf Page}\par\nobreak}
%\clearemptydoublepage
%

\iffalse
%%%%% LIST OF TABLES
\listoftables
\addtocontents{lot}{\par\nobreak\textbf{{\scshape Table} \hfill Page}\par\nobreak}
\clearemptydoublepage
%%%%% LIST OF FIGURES
\listoffigures
\addtocontents{lof}{\par\nobreak\textbf{{\scshape Figure} \hfill Page}\par\nobreak}
\clearemptydoublepage
\fi
%
%
% The bulk of the document is delegated to these chapter files in
% subdirectories.
\mainmatter
%

%\part{Prelude}

\import{chapters/}{0-introduction.tex}

\import{chapters/}{1-methods.tex}

\part{
	%\hspace{0.1cm}
	Supporting the maintenance and evolution of Coq
}

\import{chapters/}{2-pull-based.tex}

\import{chapters/}{3-bug-tracker.tex}

\import{chapters/}{4-release.tex}

\part{
	%\hspace{0.1cm}
	Organizing Coq's package ecosystem
}

\import{chapters/}{5-distribution.tex}

\import{chapters/}{6-maintenance.tex}

%\part*{Conclusions}
%\addcontentsline{toc}{part}{Conclusions}

\import{chapters/}{7-conclusion.tex}

%\clearemptydoublepage
%
%
% APPENDIX
\iffalse
\appendix
\import{chapters/appendices/}{app0A.tex}
\clearemptydoublepage
\fi


%
% Apparently the guidelines don't say anything about citations or
% bibliography styles so I guess we can use anything.
\backmatter
\bibliographystyle{siam}
\refstepcounter{chapter}
\bibliography{thesisbiblio}




%% credits for the template
\clearemptydoublepage
\newpage
\thispagestyle{empty}
\vspace*{\fill}
\centering{
\footnotesize{{\LaTeX{} thesis template adapted from \href{https://www.overleaf.com/latex/templates/university-of-bristol-thesis-template/kzqrfvyxxcdm#.W1h6jeFfiEJ}{V\'ictor F. Bre\~na-Medina's template} (Creative Commons CC BY 4.0)}

\url{https://frama.link/thesis_template}}


%\clearemptydoublepage
%
% Add index
%\printindex
%   
\end{document}
